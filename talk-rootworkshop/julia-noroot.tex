\documentclass[16pt]{beamer}
\definecolor{links}{HTML}{2A1B81}
\hypersetup{colorlinks,linkcolor=,urlcolor=links}
\usepackage{listings}
\usepackage{xcolor}
\usepackage[absolute,overlay]{textpos}
\usepackage{graphicx}

\definecolor{ForestGreen}{RGB}{60, 140, 60}

\usetheme[
	bullet=circle,		% Other option: square
	bigpagenumber,		% circled page number on lower right
	topline=true,		% colored bar at the top of the frame 
	]{Zurich}

\lstdefinelanguage{Julia}%
  {morekeywords={abstract,break,begin,case,catch,const,continue,do,else,elseif,%
      end,export,false,for,function,immutable,import,importall,if,in,%
      macro,module,otherwise,quote,return,switch,true,try,type,typealias,%
      using,while},%
   sensitive=true,%
   alsoother={$},%
   morecomment=[l]\#,%
   morecomment=[n]{\#=}{=\#},%
   morestring=[s]{"}{"},%
   morestring=[m]{'}{'},%
}[keywords,comments,strings]%

\lstset{%
    language         = Julia,
    basicstyle       = \ttfamily,
    keywordstyle     = \bfseries\color{blue},
    stringstyle      = \color{magenta},
    commentstyle     = \color{ForestGreen},
    showstringspaces = false,
}

%-----------------------------------------------------------------------------
% DOCUMENT PROPERTIES

\author[Joosep Pata]{Joosep Pata\\ \texttt{<pata@phys.ethz.ch>}}
\institute[ETHz]{ETH Zürich, Institute of Particle Physics}
\title{Julia: superglue for scientific computing}
\titlegraphic{\includegraphics[height=0.8cm]{ipp.jpeg}}

%-----------------------------------------------------------------------------


\begin{document}


% ----------------------------------------------------------------------------
\frame{

\titlepage

}
% ----------------------------------------------------------------------------


\section{Example}


\begin{frame}[fragile]
\frametitle{Motivation}

\begin{center}
Restrict problem space to numerics - speed of C, simplicity of Python.
\end{center}

\vspace{.3cm}
\begin{center}
Optimizations based on existing technology - LLVM
\end{center}

\vspace{.3cm}
\begin{center}
Data analysis now common - benefit from external developments.
\end{center}

\vspace{.3cm}
\begin{center}
This talk is about how julia can be used for scientific computing.
\end{center}


\end{frame}

% ----------------------------------------------------------------------------
\begin{frame}[fragile]
\vspace{0.5cm}
\frametitle{Why another language?}

The two-language problem
\begin{itemize}
\item static languages: great for experts, large, low-level applications, real-time
\item data analysis and exploration: \textcolor{red}{iteration}, \textcolor{ForestGreen}{experimentation}, \textcolor{gray}{unstructured}
\end{itemize}

\begin{center}
Data analysis now massively popular, a typical physicist has limited (quality) experience with C++: \textbf{just wants the result}(TM)
\end{center}

\begin{center}
Enter \textcolor{ForestGreen}{Python}, \textcolor{red}{R}: surging popularity in data analysis, science.
\end{center}

But \textbf{no} knowledge of types $\Rightarrow$ \textbf{no} fast machine code.
\begin{lstlisting}
p = Particle()
foo(p) #must do explicit type checks every time
\end{lstlisting}

%\begin{textblock*}{5cm}(5cm,7cm) % {block width} (coords)
%\includegraphics[width=2cm]{julia.png}
%\end{textblock*}

%\begin{textblock*}{5cm}(5cm,7cm) % {block width} (coords)
%\includegraphics[width=2cm]{julia.png}
%\end{textblock*}

\end{frame}
% ----------------------------------------------------------------------------

\begin{frame}[fragile]
\frametitle{What has been tried?}
\begin{center}
\Large{Vectorize and offload "heavy" stuff to a dedicated kernel?}
\end{center}

\begin{itemize}
\item Leads to "expert" and "non-expert" code
\item Artificial boundaries: user functions not callable
\item Not every problem (easily) vectorizable: \textcolor{red}{user-defined types}
\item Cost(human time) > Cost(machine time): but don't want to iterate weeks to try ideas!
\end{itemize}

\begin{center}
\textcolor{red}{Let the compiler do the hard work, write code as is most natural.}
\end{center}

\begin{columns}
\column[t]{5cm}

\centering{\textcolor{red}{vectorized}}
\begin{lstlisting}
rows = [row1, ...]
analyzeAll(rows)
\end{lstlisting}

\column[t]{5cm}
\centering{\textcolor{ForestGreen}{serial}}
\begin{lstlisting}
for (i=0;i<100;i++) {
  row = getRow(i);
  analyze(row);
}
\end{lstlisting}
\end{columns}

\end{frame}

% ----------------------------------------------------------------------------

\begin{frame}
  \frametitle{The solution}

  \vspace{.4cm}
  \begin{center}
  \Large{\textcolor{red}{High-level}, \textcolor{ForestGreen}{fast}, \textcolor{blue}{dynamically compiled} numerics.}
  \end{center}

  \begin{itemize}
  \item Started at MIT CSAIL in 2011, now open-source, worldwide activity.
  \item Easy to use (like MATLAB, R), for generic numerical computing
  \item Used for physics, bio-informatics, statistics, image processing, finance
  \item Modular design: well-tested core + packages
  \item Code and issues tracked on github, (too) easy to contribute.
  \item Based on LLVM, OpenBLAS/Intel MKL
  \end{itemize}

  \begin{textblock*}{5cm}(5cm,0.3cm) % {block width} (coords)
  \includegraphics[width=2cm]{julia.png}
  \end{textblock*}

\end{frame}

\begin{frame}[fragile]
  \frametitle{Development status}
    \begin{textblock*}{10cm}(0.5cm,1.2cm)
    Core workflow and discussions on:
    \href{https://github.com/JuliaLang/julia/}{github.com/JuliaLang/julia}
    \end{textblock*}
  \begin{textblock*}{5cm}(0.5cm,2cm) % {block width} (coords)
    \includegraphics[height=2cm]{commits.png}
  \end{textblock*}
  \begin{textblock*}{5cm}(8.5cm,2cm) % {block width} (coords)
    \includegraphics[height=2cm]{issues.png}
  \end{textblock*}
  \begin{textblock*}{10cm}(0.7cm,4.5cm)
    Daily commit activity:
  \end{textblock*}
  \begin{textblock*}{5cm}(0.5cm,5cm) % {block width} (coords)
    \includegraphics[height=2.5cm]{commits_per_day.png}
  \end{textblock*}
\end{frame}

\begin{frame}[fragile]
  \frametitle{How it looks like...}
  Download binaries for OSX/Linux/Win, run REPL:
  \begin{lstlisting}
julia> 1+2
3
julia> Pkg.update()
julia> Pkg.add("Distributions")
julia> using Distributions, ROOT
julia> h = TH1D("h", "Hist", Int32(10), -10, 10)
  \end{lstlisting}
  \begin{itemize}
    \item Run code in \textcolor{red}{batch mode} (e.g. on cluster): \texttt{julia code.jl}.
    \item Run interactive environment: \texttt{jupyter notebook} 
  \end{itemize}

  The \href{http://docs.julialang.org/en/release-0.4/manual/}{manual} is very useful.
\end{frame}


\begin{frame}[fragile]
  \frametitle{Dynamic, optional types}

  \begin{centering}

  Types may be specified or inferred automatically.
  User-side code auto-typed (time saver), library-side explicit-typed
  \begin{lstlisting}
  julia> x = 1
  1
  julia> s = "asd"
  "asd"
  julia> bla = Uint32(2)
  0x00000002
  \end{lstlisting}
  Operations with \texttt{x, s, bla} will be \textbf{machine-level}!

  Functions may be typed, compiler figures out what to do at runtime:
  \begin{lstlisting}
    julia> f(x::Int64) = x^2
    julia> g(y, z) = sqrt(y^2 + z^2)
  \end{lstlisting}
  \end{centering}
\end{frame}

\begin{frame}[fragile]
  \frametitle{Multiple dispatch}
  \begin{itemize}
  \item single dispatch: \verb|spaceship.collide(asteroid)|
  \item multiple dispatch: \verb|collide(spaceship, asteroid)|, \verb|collide(spaceship, spaceship)|
  \end{itemize}

  Natural for mathematical code (operations are global). Easy to make common APIs by extending libraries. Leverages type system instead of boxing.
  \begin{centering}
  \end{centering}
\end{frame}

\begin{frame}[fragile]
  \frametitle{Underlying code}
  What code is actually generated for a function?
  \begin{lstlisting}
  #define a new simple function
  julia> f(x) = x > 0 ? -1 : 1
  f (generic function with 2 methods)
  #check produced code if x is Int64
  julia> code_llvm(f, (Int64, ))
  #that's the low-level LLVM code
  define i64 @julia_f_21502(i64) {
  top:
    %1 = icmp slt i64 %0, 1
    %. = select i1 %1, i64 1, i64 -1
    ret i64 %.
  \end{lstlisting}
  As fast as clang/C++!
\end{frame}

\begin{frame}[fragile]
\frametitle{Speed}

\begin{center}
\textbf{fast} basic functionality: loops, floating-point operations, external C/Fortran calls
\end{center}

\textcolor{ForestGreen}{Python}:
\begin{lstlisting}
  In [11]: %time
  for i in arange(100000000):
    x+=random.random()
  Out [11] Wall time: 20.2 s
\end{lstlisting}

\textcolor{red}{julia}:
\begin{lstlisting}
  julia> @time (
  for i=1:100000000
    x += rand()
  end)
> 3.501108 seconds
\end{lstlisting}
Julia can outperform industry-standard C++ codes: \href{https://gist.github.com/skariel/3d2018f9341a058e00fc}{tesselation}.
\end{frame}

\begin{frame}[fragile]
\frametitle{Type system}

Types are simple, low-overhead and fast! Can make types based on input data at runtime.
\begin{lstlisting}
type Particle
  momentum::Float64 #explicitly specified
  name #no specified type, can be Anything
  friends::Vector{Particle} #complex type
end
my_p = Particle(0, "muon", [])
foo(p::Particle) = p.momentum^2
\end{lstlisting}

\begin{itemize}
\item Types are compiled: no overhead for e.g. \texttt{particle.momentum}
\item code is  specialized based on exact type specification information
\item \textbf{no speed difference} with respect to built-in types
\item types specify \textcolor{red}{only data}: multiple dispatch for member functions
\item Easily add additional functions to types, e.g. \texttt{foo(t::TTree)}
\end{itemize}

\end{frame}


%\begin{center}
%Compile functions \textbf{just-in-time} when types are known! 
%\end{center}

% ----------------------------------------------------------------------------

% ----------------------------------------------------------------------------
\begin{frame}[fragile]
\frametitle{Simple Mandlebrot example}

\begin{lstlisting}
function mandel(z)
    c = z
    maxiter = 80
    for n = 1:maxiter
        if abs(z) > 2
            return n-1 #asd
        end
        z = z^2 + c
    end
    return maxiter
end
\end{lstlisting}
\end{frame}
% ----------------------------------------------------------------------------


\begin{frame}
  \frametitle{Speed comparisons}
  \begin{centering}
  \includegraphics[width=12cm]{speedcomp.png}
  \end{centering}
  Naive julia implementation often similar to or better than C / Fortran, 
\end{frame}

\begin{frame}[fragile]
  \frametitle{Python interop}
  \verb|PyCall.jl|, written by Steven Johnson (fftw)

  Can import modules.
  \begin{lstlisting}
@pyimport numpy.random as nr
nr.rand(3,4)
  \end{lstlisting}
  
  Pass julia functions!
  \begin{lstlisting}
@pyimport scipy.optimize as so
so.newton(x -> cos(x) - x, 1)
  \end{lstlisting}
  
  Methods have special syntax: \verb|my_dna.find("ACT")| becomes
  \begin{lstlisting}
@pyimport Bio.Seq as s
@pyimport Bio.Alphabet as a
my_dna = s.Seq("AGTACACTGGT", a.generic_dna)
my_dna[:find]("ACT")
  \end{lstlisting}
  \begin{centering}
  \end{centering}
\end{frame}

\begin{frame}[fragile]
  \frametitle{C/Fortran interop}
  Can call C or Fortran libraries natively, no overhead.

  \begin{lstlisting}
const LIBOL = "libopenloops.so"
#id - process id (numeric)
#pp - array of particle momenta (4*N 1D)
#m2_tree - array with amplitude
function ol_evaluate_tree(id, pp, m2_tree)
  ccall(
      (:ol_evaluate_tree, LIBOL),
      Void,
      (Cint, Ptr{Cdouble}, Ptr{Cdouble}),
      Cint(id), pp, m2_tree
      )
end
ol_setparameter_int("order_ew", 1)
ol_evaluate_tree(id, pp, m2_tree)
  \end{lstlisting}
\end{frame}

\begin{frame}[fragile]
  \frametitle{C++ interop}
  From community package \href{Cxx.jl}{https://github.com/Keno/Cxx.jl}, requires development version (0.5) of julia with latest (svn) LLVM.

  \begin{lstlisting}
  using Cxx
  cxx""" #include<iostream> """  

  cxx"""
    float mycppfunction(int a, float b) {   
      int y = (int)(b * 10.0);
      int x = 10;
      return a + x*y + 2;
  }
  """
  # Convert C++ to Julia function
  julia> result = @cxx mycppfunction(1, 1.0)
  \end{lstlisting}

\end{frame}

\begin{frame}[fragile]
  \frametitle{Plotting}
  Many competing packages, matplotlib.pyplot very stable:

  \begin{center}
    \includegraphics[width=8cm]{plot.png}
  \end{center}
\end{frame}

\begin{frame}[fragile]
  \frametitle{One more thing...}
  \begin{itemize}
    \item Support for automatic SIMD vectorization: \texttt{@simd for i=1:10000 x[i]*y[i] end} $\rightarrow$ linear speed-up
    \item Built-in parallelization, coroutines: one machine to a cluster, no \texttt{GIL} like in Python, memory-shared arrays
    \item Code $=$ Data: can manipulate program code on the fly (think LISP)
    \item Interactivity through \textbf{Jupyter} kernel, one of the main foci
    \item Language interop: C, Fortran natively, all of Python; soon C++ through add-ons: can wrap a complex library in a weekend.
    \item Many mature \href{http://pkg.julialang.org/}{packages}: statistics, storage, optimization, plotting, MVAs
    \item Soon: compile directly to GPU code
  \end{itemize}
\end{frame}

\begin{frame}[fragile]
  \frametitle{Notebook interface}
  Julia supported natively in the jupyter notebook.
  
  \begin{itemize}
    \item{\href{https://github.com/jpata/JuliaForPhysicists/blob/master/tabular.ipynb}{ROOT trees}}
    \item{\href{https://github.com/jpata/JuliaForPhysicists/blob/master/Stats.ipynb}{histograms}}
    \item{\href{https://github.com/jpata/JuliaForPhysicists/blob/master/Histograms.ipynb}{statistics}}
    \item{\href{https://github.com/jpata/JuliaForPhysicists/blob/master/Cxx.ipynb}{C++ interop}}
  \end{itemize}
\end{frame}

\begin{frame}[fragile]
  \frametitle{Traction}
  \begin{textblock*}{5cm}(0.5cm,2cm) % {block width} (coords)
    \includegraphics[width=12cm]{popularity.png}
  \end{textblock*}
\end{frame}

\begin{frame}
  \frametitle{Summary}
  \begin{center}
  \Large \textcolor{red}{High-level code can still be fast.}
  \end{center}

  \begin{center}
  \Large \textcolor{gray}{Julia gaining traction in teaching, research and industry.}
  \end{center}

  \begin{center}
  \Large \textcolor{ForestGreen}{Julia used in and useful for high-energy physics}
  \end{center}

  \begin{center}
  \Large \textcolor{black}{ROOT and julia are complementary.}
  \end{center}

\end{frame}

\begin{frame}[fragile]
  \frametitle{Julia drawbacks}
  \begin{itemize}
    \item Needs relatively new compiler (gcc 4.8+) and up-to-date software stack
    \item Less feature complete than python: database interface, domain-specific libraries
    \item garbage collection introduces real-time difficulties
    \item C++ support only in unstable development version with LLVM 3.8
    \item 
  \end{itemize}
\end{frame}

\begin{frame}
\frametitle{Teaching}
Julia is used in \href{http://julialang.org/teaching/}{teaching} @ MIT since 2013.
Optimization, linear algebra, mathematical programming, numerical computation, PDEs.

\begin{itemize}
  \item \textcolor{ForestGreen}{University of Edinburgh, Spring 2016}
  \begin{itemize}
    \item MATH11146, Modern optimization methods for big data problems (Prof. Peter Richtarik)
  \end{itemize}
  
  \item \textcolor{red}{MIT, Fall 2015}
  \begin{itemize}
    \item 6.251/15.081, Introduction to Mathematical Programming (Prof. Dimitris J. Bertsimas)
    \item 18.06, Linear Algebra (Dr. Alex Townsend)
    \item 18.303, Linear Partial Differential Equations: Analysis and Numerics (Prof. Steven G. Johnson)
    \item 18.337/6.338, Numerical Computing with Julia (Prof. Alan Edelman). (IJulia notebooks)
    \item 18.085/0851, Computational Science And Engineering I (Prof. Pedro J. Sáenz)
  \end{itemize}
  
  \item \textcolor{blue}{“Sapienza” Univeristy of Rome, Italy, Spring 2015}
  \begin{itemize}
    \item Operations Research (Giampaolo Liuzzi)
  \end{itemize}

\end{itemize}
\end{frame}

\begin{frame}
\frametitle{Industry}
Several companies are working with julia and contributing to it (open-source).

\begin{itemize}
  \item Intel working on adding multi-threading to julia.
  \item Facebook working on database interfaces.
  \item Google investigating, some work on static analysis tools.
  \item Finance companies: BlackRock
\end{itemize}
\end{frame}

% ----------------------------------------------------------------------------
\begin{frame}
  \begin{textblock*}{5cm}(4cm,1.3cm)
  \Large \textcolor{black}{Thank you! Questions?}
  \end{textblock*}
  \begin{center}
    \includegraphics[width=10cm]{SaasFeeSummer.jpg}
  \end{center}
\end{frame}
\usebackgroundtemplate{}
% ----------------------------------------------------------------------------


\end{document}
